\documentclass{ip-doc}
\usepackage{lipsum}

\title{Example IP Module}
\author{Wuqiong Zhao}
\version{1.0}

\begin{document}
\maketitle

\begin{factstable}
\rowcolors{2}{gray!10}{white}
\renewcommand{\arraystretch}{1.2}
\begin{tabularx}{\linewidth}{l|X}
\toprule
  Name & Example IP \\
  Identifier & \texttt{org.wqzhao.example\_ip} \\
  Version & \theversion \\
  Author & \href{https://wqzhao.org}{Wuqiong Zhao} {\scriptsize(me@wqzhao.org)} \\
  Device & AMD UltraScale+ \\
  Platform & Vivado \\
  Design Files & Verilog \\
  Simulation Model & Verilog \\
  Constraints File & N/A \\
\bottomrule
\end{tabularx}
\end{factstable}

\section{Introduction}
The ``Example IP'' is just an example.
You will need to create your own IP module to use this class to document it elegantly.

\section{Features}

\begin{itemize}
  \item Support AXI4 interface.
  \item Another feature which is not quite related to \LaTeX.
\end{itemize}

\section{Usage}

\subsection{Text}
The text is in sans-serif font ``TeX Gyre Heros,'' a ``Helvetica'' clone.
Section numbers are in the left margin.

\subsubsection{Subsubsection}
Here is a subsubsection, the smallest sectioning command that comes with numbering.

\paragraph{Paragraph.}
Here is a paragraph, which has a run-in header.

\subsection{Math}
Math fonts are still the serif ones.
For example $a^2 + b^2 = c^2$.
Display equations can also be used:
\begin{equation}
  \int_0^1 x^2 \, \mathrm{d}x = \frac13.
\end{equation}

Here is also \lipsum

\end{document}
